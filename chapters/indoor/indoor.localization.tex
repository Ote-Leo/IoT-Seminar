\chapter{Indoor Localization}

Localization within interior contexts such as supermarkets, airports, railway stations, and hospitals has become inevitable with the rise of the \ac{iot}.
In grocery stores, customers can choose a cart equipped with a \ac{pda} screen equipped \ac{rfid} tags;
the cart's location is identified using a hybrid \ac{wifi} and \ac{rfid} system;
and once the customer wants to find the location of a product, they can browse through the \ac{pda} screen, and the directions to the target are provided\cite{kourouthanassis2001last}.
Rather than handing out pamphlets to tourists at museums around the glob, roaming tourists may be managed by a Bluetooth and \ac{wifi}-enabled gadget.
The gadget then directs visitors to a specific area of the gallery and provide information of an artefact.

Bluetooth beacons can lead students to the locations of books in libraries.
The student might use a mobile device to access his position to the network and recommend him.
The precision of the position is within meters, ensuring that the scholar is close to the relevant book shelf\cite{hahn2017indoor}.
It is crucial in clinical institutions to supervise some patients' movements, primarily those suffering from mental impairments such as Alzheimer's; \ac{rfid} technology can help with this\cite{calderoni2015indoor}.
\Ac{rfid} tags may also be implanted on various areas of the body, allowing patients who require home-healthcare to know if they are sitting, sleeping, walking, standing, or collapsing, which demands close monitoring and prompt response\cite{shuaieb2020rfid}.

\section{Localization Systems Technologies}

\subsection{Satellite-Based Navigation}
The most widely used technology for outdoor localization is the \ac{gps}.
It does, however, necessitate \acf{los} between the satellites and the handset.
As a consequence, primarily caused by external walls of the building, it has become ineffective for interior location-based service.
The \ac{gps} may be used as the front-end of an \ac{gps} receiver by utilizing a steerable, high gain directional antenna.
Pseudolites (i.e., pseudo satellites) are deployed as independent localization systems in areas where \ac{gps} signals cannot be received;
these systems are comprised of pseudolites, transmitter and receiver antennas, target receivers, and reference.
The basic idea is to receive the \ac{gps} signal and rebroadcast it via indoor transmitters\cite{xu2015new}.

\subsection{\acf{ins}}
\ac{imu} such as accelerometers and gyroscopes are used in \ac{ins}s to define the location and directional movement of an item to an initial location, velocity, and angle.
\ac{ins} is recognized by its precision and resource utilization given that the inertial sensor is mounted to the surface of the object.
However, \ac{ins} may become vulnerable to distortion, necessitating the use of specialized filtering systems such as the Kalman filter\cite{hu2020improving}.
Another disadvantage of employing \ac{ins} is the expense and effort required to build the location sensor's network infrastructure.
A unique initial location estimation approach has been proposed in \cite{chen2016smartphone} by integrating current \ac{wifi} routers and iBeacons.

\subsection{Magnetic Based Navigation}
At low altitudes, magnetic-based technologies are used for localization.
A magnetic sensor receives the radiated fields from at least three reference magnetic stations, and the sensor's placement is computed via trilateration.
At low frequencies, this technique is accurate;
nonetheless, it is sensitive to conductive and ferromagnetic materials.
Magnetic-based navigation systems often rely on disruptions in the Earth's magnetic field within surroundings;
these disturbances arise owing to the ferromagnetic nature of metal structures within buildings\cite{shu2015magicol}.

Magnetic maps are developed by measuring the magnetic magnetic field at known places;
these maps can estimate the position of an unknown target based on its magnetic measurement.
This method is known as magnetic fingerprinting.
Magnetic interference, on the other hand, might be severe and produce localization issues.
Using multiple handsets along the same routes may result in different magnetic readings being recorded.
The authors of \cite{lee2018amid} investigated localization using deep learning;
accuracy was determined to be $0.8 m$ in corridors and $2.3 m$ in the atrium.
Cameras and magnetic fields are integrated with neural networks to improve localization when the phone is upright;
their proposed techniques \cite{liu2016fusion} achieved more than 91\% percent accuracy increase of $1.34 m$.

\subsection{\acf{rf} Based Navigation}
The most widely used techniques for localization are those based on radio frequencies.
It is advocated since it covers a larger territory with low-cost gear.
This is reinforced by the fact that \ac{rf} waves can permeate solids such as walls and human bodies.
\ac{rf}-based navigation technologies outperform alternative localization techniques such as infrared and ultrasonic navigation systems.
These systems, however, should be avoided in hospitals and aircraft since they may conflict with current \ac{rf} systems.

Since the radio frequency is less than $300 GHz$, wireless technologies employed for interior localization can be classified.
At the same time, the frequency of wireless technology determines its capabilities such as coverage, wall penetration, and resistance to barriers.
Thus, there are three primary categories of wireless technology for diverse applications: long-distance, medium-distance, and short-distance technology.

In a \acf{wsn}, node position information is critical for several functions such as routing, clustering, and context-based applications.
\ac{wsn} is described as a network of nodes that detect the environment's fields and wirelessly communicate the acquired data\cite{reichenbach2006indoor}.
These details are sent to the sink node, which is used to gather data.
In \ac{wsn}, knowledge about the position of nodes is vital since observations lacking position information are pointless.

An examples of localization in \ac{wsn} is demonstrated using \ac{rssi} based on the ZigBee standard\cite{mankong20205}.
\ac{wsn} may employ both range-based and free-range for localization\cite{munadhil2020neural}.
It is further classified as accurate (lateration, trilateration) and approximate.
\ac{wifi}, Bluetooth, ZigBee, \acf{uwb}, and \acf{rfid} are a few examples of \ac{rf}-based navigation systems.

\subsubsection{\ac{wifi} Technology}
Most large communal environments, such as colleges or business buildings, have already deployed \ac{wifi} hotspots that offer network access point coverage throughout the building.
\ac{wifi} technology is used in a variety of devices, including personal computers, video game consoles, cellphones, digital cameras, tablet computers, and digital music players.
That infrastructure cost may be quite minimal, and \ac{wifi} has expanded from a reception range of roughly $100 m$ to about $1 km$.
Furthermore, \ac{wifi} localization based on fingerprinting \ac{rss} might be used in conjunction with other \ac{rf} localization methods, such as \ac{rfid}\cite{wang2016indoor}.
\ac{wifi} covers a larger area than Bluetooth and has a higher throughput, making it more practical to use.
% TODO: find the citation for HORUS and COMPASS
RADAR\cite{bahl2000radar}, HORUS\cite{bahl2000radar}, and COMPASS\cite{bahl2000radar} are examples of commercial \ac{wifi}-based navigation systems.

\subsubsection{ZigBee}
ZigBee is an IEEE 802.15.4 standard-based specification.
It operates in the $868 MHz$ band in Europe, the $915 MHz$ band in the United States and Australia, and the $2.4 GHz$ spectrum in other locations.
ZigBee is a wireless mesh network protocol that allows for long-distance communication between devices.
When compared to \ac{wifi} standards, it has a low cost, a modest data transmission rate, and a short latency time.
The \ac{rss} technique is used by the technology to quantify the distance between two or more ZigBee sensor devices\cite{sugano2006indoor}.
The scanning of \ac{ap}s via the \ac{wifi} interface consumes a lot of electricity.
To mitigate this impact, the author in \cite{niu2015zil} proposed ZIL, an energy-efficient indoor localization utilizing ZigBee in which the ZigBee interface is utilized to gather \ac{wifi} signals.

\subsubsection{Bluetooth}
Bluetooth is created to encourage devices to communicate wirelessly across short distances.
Bluetooth uses radio waves with frequencies ranging from $2.402 GHz$ to $2.480 GHz$ to communicate.
If it has qualities such as low transmission power, battery life, secure and efficient communications, and an easily accessible solution.
The new \ac{ble} can cover a range of $70-100 m$ and offer $24Mbps$ while using less power\cite{zafari2019survey}.
As a result, Bluetooth is unsuitable for large-area localization.
Using the \ac{rss} trained \ac{nn} trained in \cite{Altini2010BluetoothIL}, and they may be used to detect user location based on online \ac{rss} measurements.

\subsubsection{\acf{rfid}}
\acf{rfid} systems rely on \ac{rfid} tags' back-scattering communication, as well as \ac{rfid} readers and middleware to process the signal created between the tags and the readers\cite{tesoriero2010improving}.
\ac{rfid} tags are classified as either active, passive, or semi-active.
Active \ac{rfid} tags have a built-in battery included within their electronics.
Active \ac{rfid}s operate at \ac{uhf} and \ac{shf}, with detection spans of $100 m$.
Consequently, active \ac{rfid} may be used for long-range localization and object tracking\cite{deak2012survey}.
Active \ac{rfid} technology, on the other hand, is unreliable for sub-meter localization precision and is not widely available on most portable user devices.
Passive tags lack internal batteries and instead reflect the signal received from the base station.
Passive \ac{rfid} is widely utilized for a variety of applications owing to its multiple advantages, including low cost, reduced size, and ease of manufacture compared to active \ac{rfid}, which requires only a tag chip and an antenna.
Passive \ac{rfid} is helpful for sub-meter detections and has a detection range of $10 m$\cite{deak2012survey}.

% \section{Localization Detection Techniques}

% \subsection{Proximity Based Technique}

% \subsection{Scene Analysis}

% \subsection{Triangulation}

% \subsubsection{Lateration}

% \subsubsection{Angulation}

% \section{Localization Methods \& Algorithms}

% \subsection{\acf{aoa} Measurements}

% \subsection{\acf{toa} Measurements}

% \subsection{\acf{rss}}

% \subsection{Radio-Frequency Fingerprinting}

\begin{sidewaystable}
    \tiny
    \centering
    \caption{Localization Techniques Performance Overview}
    \begin{tabular}{ l l l l l l l l }
        \hline
        Technology     & Technique                  & Method    & Accuracy (m)   & Cost       & Coverage                & Pros                             & Cons                \\
        \hline
        Satellite      & Trilateration              & \ac{toa}  & $3-5$          &            & Floor level             & Lower power consumption          &                     \\
                       &                            & TDOA      &                &            &                         &                                  &                     \\
                       &                            &           &                &            &                         &                                  &                     \\
        Inertial       & \ac{dr}                    & ---       & 2              & Low        & Floor level             & Cheap                            & Accumulative errors \\
                       &                            &           &                &            &                         &                                  &                     \\
        Magnetic based & Trilateration              & ---       & 2              & Low        & Floor level             & Cheap                            & Requires mapping    \\
                       &                            &           &                &            &                         &                                  &                     \\
        \ac{wifi}      & Proximity                  & \ac{ap}ID & 10 (proximity) & Low        & Floor level (around 35) & \multirow{6}{12em}{Good accuracy                       \\ Low cost \\ \ac{wifi} signal can penetrate \\ walls \\ No need for additional \\ infrastructure} & \multirow{4}{12em}{\ac{rf} interface with devices \\ operating at $2.4GHz$ \\ Fingerprinting requires a \\ huge effort} \\
                       & Trilateration              & \ac{rss}  & $1-5$          &            &                         &                                  &                     \\
                       & Angulation                 & \ac{toa}  &                &            &                         &                                  &                     \\
                       & Fingerprinting             & TDOA      &                &            &                         &                                  &                     \\
                       & \ac{rss}-Propagation Model & \ac{aoa}  &                &            &                         &                                  &                     \\
                       &                            &           &                &            &                         &                                  &                     \\
                       &                            &           &                &            &                         &                                  &                     \\
        ZigBee         & Proximity                  & \ac{ap}ID & $3-5$          & Medium     & Floor level             & \multirow{2}{12em}{Low Cost                            \\ Low power consumption} &  Requires special equipment \\
                       & Trilateration              & \ac{rss}  &                &            &                         &                                  &                     \\
                       & Fingerprinting             &           &                &            &                         &                                  &                     \\
                       & \ac{rss}-Propagation Model &           &                &            &                         &                                  &                     \\
                       &                            &           &                &            &                         &                                  &                     \\
        Bluetooth      & Proximity                  & \ac{ap}ID & $2-5$          & Low-Medium & around 10               & \multirow{3}{12em}{Good accuracy                       \\ No need for additional \\ infrastructure \\ Low power consumption} &  \multirow{3}{12em}{\ac{rf} interface \\ Limited coverage \\ and  mobility} \\
                       & Trilateration              & \ac{rss}  &                &            &                         &                                  &                     \\
                       & Fingerprinting             & \ac{toa}  &                &            &                         &                                  &                     \\
                       &                            &           &                &            &                         &                                  &                     \\
                       &                            &           &                &            &                         &                                  &                     \\
        \ac{rfid}      & Proximity                  & \ac{ap}ID & $1-5$          & Low        & Room level              & \multirow{2}{12em}{Cheap                               \\ Realtime localization} &  \multirow{2}{12em}{Low accuracy \\ Response time is high} \\
                       & Trilateration              & \ac{rss}  &                &            &                         &                                  &                     \\
                       & Fingerprinting             &           &                &            &                         &                                  &                     \\
                       & \ac{rss}-Propagation Model &           &                &            &                         &                                  &                     \\
                       &                            &           &                &            &                         &                                  &                     \\
        \hline
    \end{tabular}
\end{sidewaystable}